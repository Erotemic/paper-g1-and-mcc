\documentclass{article}

\usepackage{arxiv}

%\usepackage[utf8]{inputenc} % allow utf-8 input
\usepackage[T1]{fontenc}    % use 8-bit T1 fonts
\usepackage{hyperref}       % hyperlinks
\usepackage{url}            % simple URL typesetting
\usepackage{booktabs}       % professional-quality tables
\usepackage{amsfonts}       % blackboard math symbols
\usepackage{nicefrac}       % compact symbols for 1/2, etc.
\usepackage{microtype}      % microtypography
\usepackage{lipsum}         % Can be removed after putting your text content
\usepackage{graphicx}
\usepackage[numbers]{natbib}
\usepackage{doi}

%% extra
\usepackage{listings}
\usepackage{amsmath} 
\usepackage{xcolor}

\usepackage{cleveref}       % smart cross-referencing


\definecolor{codegreen}{rgb}{0,0.6,0}
\definecolor{codegray}{rgb}{0.5,0.5,0.5}
\definecolor{codepurple}{rgb}{0.58,0,0.82}
\definecolor{backcolour}{rgb}{0.95,0.95,0.92}

\lstdefinestyle{mystyle}{
    backgroundcolor=\color{backcolour},   
    commentstyle=\color{codegreen},
    keywordstyle=\color{magenta},
    numberstyle=\tiny\color{codegray},
    stringstyle=\color{codepurple},
    basicstyle=\ttfamily\footnotesize,
    breakatwhitespace=false,         
    breaklines=true,                 
    captionpos=b,                    
    keepspaces=true,                 
    numbers=left,                    
    numbersep=5pt,                  
    showspaces=false,                
    showstringspaces=false,
    showtabs=false,                  
    tabsize=2
}

\definecolor{light-gray}{gray}{0.95}
\newcommand{\code}[1]{\colorbox{light-gray}{\texttt{#1}}}
\lstset{style=mystyle}


\definecolor{kwgreen}{HTML}{3EAE2B}
\definecolor{kwblue}{HTML}{0068C7}
\definecolor{kworange}{HTML}{EF7724}
\definecolor{kwred}{HTML}{F42836}
\definecolor{kwdarkgreen}{HTML}{2E5524}
\definecolor{kwdarkblue}{HTML}{003765}


\newcommand{\TP}[1]{\textcolor{kwgreen}{\texttt{\textbf{TP}}}}
\newcommand{\FP}[1]{\textcolor{kwred}{\texttt{\textbf{FP}}}}
\newcommand{\TN}[1]{\textcolor{kwblue}{\texttt{\textbf{TN}}}}
\newcommand{\FN}[1]{\textcolor{kworange}{\texttt{\textbf{FN}}}}

\newcommand{\PPV}[0]{\texttt{\textbf{PPV}}}
\newcommand{\TPR}[0]{\texttt{\textbf{TPR}}}
\newcommand{\MCC}[0]{\texttt{\textbf{MCC}}}
\newcommand{\Fowlkes}[0]{\texttt{\textbf{FM}}}
\newcommand{\Fone}[0]{\texttt{\textbf{F1}}}



% Based on my blog post
% https://erotemic.wordpress.com/2019/10/23/closed-form-of-the-mcc-when-tn-inf/

\title{
    The MCC approaches the geometric mean of precision and recall as true negatives approach infinity.
%    The limit of the MCC as the number of True negatives approaches infinity is the geometric mean of precision and recall
%    The limit of the MCC as true negatives approach infinity is the geometric mean of precision and recall.
}

\author{\href{https://orcid.org/0009-0008-8455-7514}{\includegraphics[scale=0.06]{orcid.pdf}\hspace{1mm}Jon Crall}\thanks{https://github.com/Erotemic/ \texttt{erotemic@gmail.com}} \\
	Kitware Inc.\\
	\texttt{jon.crall@kitware.com}
}

% Uncomment to override  the `A preprint' in the header
%\renewcommand{\headeright}{Technical Report}
%\renewcommand{\undertitle}{Technical Report}
\renewcommand{\shorttitle}{\textit{arXiv} Template}

%%% Add PDF metadata to help others organize their library
%%% Once the PDF is generated, you can check the metadata with
%%% $ pdfinfo template.pdf
\hypersetup{
pdftitle={The MCC approaches the geometric mean of precision and recall as true negatives approach infinity.},
pdfsubject={q-cs.CV},
pdfauthor={Jon Crall},
pdfkeywords={Confusion Matrix, Binary Classification, Fowlkes--Mallows Index, Matthew's Correlation Coefficient, F1},
}

\begin{document}
\maketitle

\begin{abstract}
    %This paper proves $\lim_{\TN{} \to \infty} \MCC{} = \Fowlkes{}$.

    The performance of a binary classifier is described by a confusion matrix with four entries:
    the number of true positives (\TP{}), true negatives (\TN{}), false positives (\FP{}), and false
      negatives (\FN{}).

    The Matthew's Correlation Coefficient (MCC), F1, and Fowlkes--Mallows (FM) scores are scalars that
      summarize a confusion matrix.
    Both the F1 and FM scores are based on only three of the four entries in a confusion matrix (they
      ignore \TN{}).
    In contrast, the MCC takes into account all four entries of a confusion matrix and thus can be seen as
      providing a more representative picture.

    However, in object detection problems, measuring the number of true negatives is so large it is often
      intractable.
    Thus we ask, what happens to the MCC as the number of true negatives approaches infinity?
    This paper provides insight into the relationship between the MCC and FM score by proving that the
      FM-measure is equal to the limit of the MCC as the number of true negatives approaches infinity.

\end{abstract}


% keywords can be removed
\keywords{Confusion Matrix \and Binary Classification \and Fowlkes--Mallows Index \and Matthew's Correlation Coefficient \and F1}


\section{Introduction}

Evaluation of binary classifiers is central to the quantitative analysis of
machine learning models \cite{powers_evaluation_2011}.
%This paper will follow the notation in the Wikipedia article on the Confusion Matrix~\cite{wiki_cfsn} as of 2023-04-30.
Given a finite set of examples with known real labels, the quality of a set of
corresponding predicted labels can quantified using a $2 \times 2$ confusion
matrix.  A confusion matrix counts the number of true positives (\TP{}), true
negatives (\TN{}), false positives (\FP{}), and false negatives (\FN{}) a model
predicts with respect to the real labels. A confusion matrix is written as:

\begin{equation}
\begin{bmatrix}
    \TP{} & \FP{} \\
    \FN{} & \TN{} 
\end{bmatrix}	
\end{equation}

This matrix provides a holistic view of classifier quality, however, it is
often desirable to summarize performance using fewer numbers. Two popular
metrics defined on a classification matrix are precision and recall.

Precision --- also known as the positive-predictive-value (PPV) --- is the
fraction of positive predictions that are correct.

\begin{equation}
    \PPV{} = \frac{\TP{}}{\TP{} + \FP{}}
\end{equation}

Recall --- also known as the true positive rate (TPR), sensitivity, or
probability of detection (PD) --- is the fraction of real positive cases that
are correct.

\begin{equation}
    \TPR{} = \frac{\TP{}}{\TP{} + \FN{}}
\end{equation}

% https://en.wikipedia.org/wiki/F-score
One of the most popular confusion metrics is the F1 score. 
It can be defined as the harmonic mean of precision and recall.

\begin{equation}
    \Fone{} = \frac{2 \PPV{} \cdot \TPR{}}{\PPV{} + \TPR{}} = \frac{2 \TP{}}{2 \TP{} + \FP{} + \FN{}}
\end{equation}

% https://en.wikipedia.org/wiki/Fowlkes%E2%80%93Mallows_index
A similar, but less used metric is the Fowlkes--Mallows index
\cite{fowlkes_method_1983}, which was originally developed for measuring the
similarity between two clusterings of a set of points.
It can be defined as the geometric mean of precision and recall
\cite{tharwat_classification_2020}.

\begin{equation}
    \Fowlkes{} = \sqrt{\PPV{} \cdot \TPR{}} = \sqrt{\frac{\TP{}}{\TP{} + \FP{}} \frac{\TP{}}{\TP{} + \FN{}}}
\end{equation}

In \cite{powers_evaluation_2011}, Powers notes that the F1 score (and
consequentially any metric that only includes precision and recall) only takes
into account three of the four measures in a confusion matrix. Powers,
introduces modifications of precision and recall he refers to as informedness
and markedness. Additionally he advocates for the use of the MCC over the F1
measure.

The Matthews Correlation Coefficient (MCC) \cite{matthews_comparison_1975} accounts for all four terms in the confusion matrix and is defined as:

\begin{equation}
    \MCC{} = \frac{
        \TP{} \cdot \TN{} - \FP{} \cdot \FN{}
    }
    {\sqrt{
        (\TP{} + \FP{}) (\TP{} + \FN{}) (\TN{} + \FP{}) (\TN{} + \FN{})
    }}
\end{equation}

While the MCC is a desirable measure due to its balanced inclusion of all terms in a confusion matrix, it
  requires that the number of true negatives is measurable.
In the case of object detection problems \cite{zou2023object}, this is often intractable as the number of
  the number of predicted boxes and missed true boxes is dwarfed by the total number of boxes that the system
  correctly did not predict.
One can see this by considering the set of all $N\times M$ boxes centered at each pixel, most of which will
  be considered true negatives.
If the width and height of the boxes are allowed to extend outside the image, then the number of predictable
  boxes actually is unbounded (and even if they must be contained in the image, there will still be a very
  large number of them in real world cases).


Because calculating the number of true negatives is difficult for open-world problems like object detection,
  it is conceptually simpler to ignore true negatives and simply focus on the much smaller set of true
  positives, false positives, and false negatives, which can be used to compute PPV, TPR, F1, and FM.
While these measures have proven themselves to be effective, simply ignoring true negatives is somewhat
  unsatisfying.
We seek to remedy this noting that in these open-world problems the number of true negatives is so large it
  is effectively infinite and thus we ask the question:
what happens to the MCC as the number of true negatives approaches infinity?

The main contribution of this paper is to highlight a relationship between the MCC and the FM score.
The MCC reduces to FM as the number of true negatives approaches infinity.
While this is not a difficult result to show, to the best of the author's knowledge, this was first shown in
  a blog post \cite{mcc_blog}, but has not yet been published.
This paper is a more formal description of this result. Specifically, the
contributions are:

\begin{itemize}

    \item We informally (but rigorously) prove the statement $\lim_{\TN{} \to \infty} \MCC{} = \Fowlkes$.

    %\item We formally prove the statement $\lim_{\TN{} \to \infty} \MCC{} = \Fowlkes$ in lean4.

    %\lim_{TN → ∞} MCC = FM$

\end{itemize}



\section{The Relationship Between MCC and FM}
\label{sec:headings}

\paragraph{Taking the limit of the MCC}

Consider the limit of the MCC as the number of true negatives approaches infinity.

\begin{equation}
    \lim_{\TN{} \to \infty} \MCC{} = \lim_{\TN{} \to \infty}
    \frac{
        \TP{} \cdot \TN{} - \FP{} \cdot \FN{}
    }
    {\sqrt{
        (\TP{} + \FP{}) (\TP{} + \FN{}) (\TN{} + \FP{}) (\TN{} + \FN{})
    }}
\end{equation}

%%% Thanks to Lee Newberg for the cleaner formulation of this limit without L'Hôpital's rule!
We can take this limit by applying some algebra to the body of the limit. We multiply the numerator and denominator by $\frac{1}{\TN{}}$:

\begin{equation}\label{eq:lee_step1}
    = \lim_{\TN{} \to \infty}
    \frac{ 
        \frac{1}{\TN{}} (\TP{} \cdot \TN{} - \FP{} \cdot \FN{}) 
    }
    {\frac{1}{\TN{}} \sqrt{
        (\TP{} + \FP{}) (\TP{} + \FN{}) (\TN{} + \FP{}) (\TN{} + \FN{})
    }} 
\end{equation}

We distribute the $\frac{1}{\TN{}}$ term in the numerator and denominator:

\begin{equation}\label{eq:lee_step2}
    = \lim_{\TN{} \to \infty}
    \frac{ 
        (\TP{} - \FP{} \cdot \frac{\FN{}}{\TN{}}) 
    }
    {\sqrt{
        (\TP{} + \FP{}) (\TP{} + \FN{}) (\frac{\TN{} + \FP{}}{\TN{}}) (\frac{\TN{} + \FN{}}{\TN{}})
    }}
\end{equation}

The $\frac{\TN{}}{\TN{}}$ terms in the denominator cancel:

\begin{equation}\label{eq:lee_step3}
    = \lim_{\TN{} \to \infty}
    \frac{ 
        (\TP{} - \FP{} \cdot \frac{\FN{}}{\TN{}}) 
    }
    {\sqrt{
        (\TP{} + \FP{}) (\TP{} + \FN{}) (1 + \frac{\FP{}}{\TN{}}) (1 + \frac{\FN{}}{\TN{}})
    }}
\end{equation}

The terms involving $\TN{}$ are fractions of simple rational polynomials
(w.r.t. $\TN{}$) and in each case the degree of the denominator is greater
than that of the numerator, so in the limit each of these terms simplifies to
$0$. Thus, the entire equation simplifies to:

\begin{equation}\label{eq:postlimit}
    = 
    \frac{ 
        (\TP{} - \FP{} \cdot 0) 
    }
    {\sqrt{
        (\TP{} + \FP{}) (\TP{} + \FN{}) (1 + 0) (1 + 0)
    }}
\end{equation}

%We can take the limit of this equation using using L'Hôpital's rule --- i.e.
%the limit is equal to the limit of the derivative of the numerator with respect
%to \TN{} divided by the limit of the denominator with respect to \TN{}.

%\begin{equation}
%    = \lim_{\TN{} \to \infty}
%    \frac{
%        \frac{\partial \TP{} \cdot \TN{} - \FP{} \cdot \FN{}}{\partial \TN{}} 
%    }
%    {
%        \frac{\partial \sqrt{(\TP{} + \FP{}) (\TP{} + \FN{}) (\TN{} + \FP{}) (\TN{} + \FN{})}}{\partial \TN{}}
%    } 
%\end{equation}

%The derivative of the numerator simplifies to $\TP{}$ by applying the product
%rule. The derivative of the denominator can be taken using the chain rule and
%product rules. Explicitly showing this process is not too difficult, but it is
%involved. For conciseness we omit the explicit steps and compute the derivative
%symbolically using a computer. Taking the derivative of the numerator and
%denominator we get:

%\begin{equation}
%    = \lim_{\TN{} \to \infty}
%    \frac{
%        \TP{} 
%    }
%    {
%        \sqrt{
%            \frac{
%                (\FN{} + \TN{}) (\FN{} + \TP{}) (\FP{} + \TP{})
%            }{
%                4 (\FP{} + \TN{})
%            }} + 
%        \sqrt{
%            \frac{
%                (\FP{} + \TN{}) (\FN{} + \TP{}) (\FP{} + \TP{})
%            }{
%                4 (\FN{} + \TN{})
%            }
%        }
%    }
%\end{equation}

%In the limit the terms with \TN{} cancel, resulting in:


%\begin{equation}
%    =
%    \frac{
%        \TP{} 
%    }
%    {
%        \frac{
%            \sqrt{(\FN{} + \TP{}) (\FP{} + \TP{})}
%        }{
%            2 
%        } + 
%        \frac{
%            \sqrt{(\FN{} + \TP{}) (\FP{} + \TP{})}
%        }{
%            2 
%        }
%    } 
%\end{equation}

Thus we find that the limit of
the MCC as true negatives approach infinity is:

\begin{equation}
    = 
    \frac{\TP{}}
    {\sqrt{
        (\TP{} + \FP{}) (\TP{} + \FN{}) 
    }}
\end{equation}

\paragraph{Rearranging the FM}

Now rearranging the equation for FM, we find it is equivalent to the limit of the MCC as the number of true negatives approaches infinity.

\begin{align}
    \Fowlkes{} &= \sqrt{\frac{\TP{}}{\TP{} + \FP{}} \frac{\TP{}}{\TP{} + \FN{}}} \\
               &= \sqrt{\frac{\TP{}^2}{(\TP{} + \FP{}) (\TP{} + \FN{})}} \\
               &= \frac{\TP{}}{\sqrt{(\TP{} + \FP{}) (\TP{} + \FN{})}} \\
               &= \lim_{\TN{} \to \infty} \MCC{}
\end{align}
%\end{equation}

\paragraph{Verifying the proof}

The correctness of these claims can be verified using SymPy \cite{sympy17}. We
define a symbolic expression for the definition of the MCC and FM score. We
then use SymPy to determine the limit of the MCC as $\TN{} \to \infty$. 
Finally we subtract expressions that we claim are equal, which will result in
zero only if they are equal.


\begin{lstlisting}[language=Python]

from sympy import sqrt, symbols, simplify
from sympy.series import limit

tp, tn, fp, fn = symbols("tp tn fp fn",
                         integer=True, negative=False)

# The definition of the MCC
numer = (tp * tn - fp * fn)
denom = sqrt((tp + fp) * (tp + fn) * (tn + fp) * (tn + fn))
mcc = numer / denom

# The definition of FM
FM = sqrt((tp / (tp + fn)) * (tp / (tp + fp)))

# Compute the limit of the MCC definition
mcc_lim = limit(mcc, tn, float('inf'))

# We claim the limit of the MCC and the FM are equivalant to:
mcc_lim_claim = tp / sqrt((tp + fn) * ((tp + fp)))

# Check the claim is equal to FM
assert simplify(FM - mcc_lim_claim) == 0
# Check the claim is equal to the MCC limit
assert simplify(mcc_lim - mcc_lim_claim) == 0

\end{lstlisting}

The above program does not raise an AssertionError, thus we have proven
$\lim_{\TN{} \to \infty} \MCC{} = \Fowlkes{}, \square{}$.


\section{Conclusion}

This paper proves that the limit of the MCC as the number of true negatives
goes to infinity is equivalent to the Fowlkes--Mallows index (i.e. the
geometric mean of precision and recall).

This is a useful insight in open world problems where the number of true negative cases grows faster than
  the number of other confusion categories.
It validates the use of precision and recall as a way of describing the quality of object detection results
  and hints that the FM score may be a preferable alternative to the more standard F1 score.


\section{Acknowledgements}

Thanks to Lee Newberg for the cleaner formulation of \cref{eq:lee_step1,eq:lee_step2,eq:lee_step3} to achieve \cref{eq:postlimit} without needing L'Hôpital's rule.

\bibliographystyle{unsrtnat}
\bibliography{references}  %%% Uncomment this line and comment out the ``thebibliography'' section below to use the external .bib file (using bibtex) .

\end{document}
