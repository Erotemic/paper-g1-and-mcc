\documentclass{article}

\usepackage{arxiv}

\usepackage[T1]{fontenc}    % use 8-bit T1 fonts
%\usepackage[utf8]{inputenc} % allow utf-8 input
\usepackage{hyperref}       % hyperlinks
\usepackage{url}            % simple URL typesetting
\usepackage{booktabs}       % professional-quality tables
\usepackage{amsfonts}       % blackboard math symbols
\usepackage{nicefrac}       % compact symbols for 1/2, etc.
\usepackage{microtype}      % microtypography
\usepackage{lipsum}         % Can be removed after putting your text content
\usepackage{graphicx}
\usepackage[numbers]{natbib}
\usepackage{doi}

%% extra
\usepackage{listings}
\usepackage{listingsutf8} % <-- key

\usepackage{amsmath} 
\usepackage{xcolor}
\usepackage{color}

\usepackage{cleveref}       % smart cross-referencing

\usepackage{amsmath,amssymb}
\usepackage{alltt}

\newcommand{\R}{\mathbb{R}}
\newcommand{\nhds}{\mathcal{N}}


\definecolor{keywordcolor}{rgb}{0.0,0.0,0.6}
\definecolor{codegreen}{rgb}{0,0.6,0}
\definecolor{codegray}{rgb}{0.5,0.5,0.5}
\definecolor{codepurple}{rgb}{0.58,0,0.82}
\definecolor{backcolour}{rgb}{0.95,0.95,0.92}

\lstdefinestyle{mystyle}{
    backgroundcolor=\color{backcolour},   
    commentstyle=\color{codegreen},
    keywordstyle=\color{keywordcolor},
    numberstyle=\tiny\color{codegray},
    stringstyle=\color{codepurple},
    basicstyle=\ttfamily\footnotesize,
    breakatwhitespace=false,         
    breaklines=true,                 
    captionpos=b,                    
    keepspaces=true,                 
    numbers=left,                    
    numbersep=5pt,                  
    showspaces=false,                
    showstringspaces=false,
    showtabs=false,                  
    tabsize=2,
    morekeywords={assert},
}

\definecolor{light-gray}{gray}{0.95}
\newcommand{\code}[1]{\colorbox{light-gray}{\texttt{#1}}}
\lstset{style=mystyle}


\definecolor{kwgreen}{HTML}{3EAE2B}
\definecolor{kwblue}{HTML}{0068C7}
\definecolor{kworange}{HTML}{EF7724}
\definecolor{kwred}{HTML}{F42836}
\definecolor{kwdarkgreen}{HTML}{2E5524}
\definecolor{kwdarkblue}{HTML}{003765}


\newcommand{\TP}[1]{\textcolor{kwgreen}{\texttt{\textbf{TP}}}}
\newcommand{\FP}[1]{\textcolor{kwred}{\texttt{\textbf{FP}}}}
\newcommand{\TN}[1]{\textcolor{kwblue}{\texttt{\textbf{TN}}}}
\newcommand{\FN}[1]{\textcolor{kworange}{\texttt{\textbf{FN}}}}

\newcommand{\PPV}[0]{\texttt{\textbf{PPV}}}
\newcommand{\TPR}[0]{\texttt{\textbf{TPR}}}
\newcommand{\MCC}[0]{\texttt{\textbf{MCC}}}
\newcommand{\Fowlkes}[0]{\texttt{\textbf{FM}}}
\newcommand{\Fone}[0]{\texttt{\textbf{F1}}}



% Based on my blog post
% https://erotemic.wordpress.com/2019/10/23/closed-form-of-the-mcc-when-tn-inf/

\title{
    The MCC approaches the geometric mean of precision and recall as true negatives approach infinity.
%    The limit of the MCC as the number of True negatives approaches infinity is the geometric mean of precision and recall
%    The limit of the MCC as true negatives approach infinity is the geometric mean of precision and recall.
}

\author{\href{https://orcid.org/0009-0008-8455-7514}{\includegraphics[scale=0.06]{orcid.pdf}\hspace{1mm}Jon Crall}\thanks{https://github.com/Erotemic/ \texttt{erotemic@gmail.com}} \\
	Kitware Inc.\\
	\texttt{jon.crall@kitware.com}
}

% Uncomment to override  the `A preprint' in the header
%\renewcommand{\headeright}{Technical Report}
%\renewcommand{\undertitle}{Technical Report}
\renewcommand{\shorttitle}{\textit{arXiv} Template}

%%% Add PDF metadata to help others organize their library
%%% Once the PDF is generated, you can check the metadata with
%%% $ pdfinfo template.pdf
\hypersetup{
pdftitle={The MCC approaches the geometric mean of precision and recall as true negatives approach infinity.},
pdfsubject={q-cs.CV},
pdfauthor={Jon Crall},
pdfkeywords={Confusion Matrix, Binary Classification, Fowlkes--Mallows Index, Matthews Correlation Coefficient, F1},
}

\begin{document}
\maketitle

\begin{abstract}
    %This paper proves $\lim_{\TN{} \to \infty} \MCC{} = \Fowlkes{}$.

    The performance of a binary classifier is described by a confusion matrix with four entries:
    the number of true positives (\TP{}), true negatives (\TN{}), false positives (\FP{}), and false
      negatives (\FN{}).

    The Matthews Correlation Coefficient (MCC), F1, and Fowlkes--Mallows (FM) scores are scalars that
      summarize a confusion matrix.
    Both the F1 and FM scores are based on only three of the four entries in a confusion matrix (they
      ignore \TN{}).
    In contrast, the MCC takes into account all four entries of a confusion matrix and thus can be seen as
      providing a more representative picture.

    However, in object detection problems, the number of true negatives is effectively unbounded.
    Thus we ask, what happens to the MCC as the number of true negatives approaches infinity?
    This paper provides insight into the relationship between the MCC and FM score by proving that the
      FM-measure is equal to the limit of the MCC as the number of true negatives approaches infinity.
    \emph{Update 2026--01--28:} This proof was previously published in ecology literature in terms of the phi
      coefficient and the Ochiai similarity, but we discuss it in the context of binary classifiers.
    We formalize the proof in Lean using AI-assistance and also comment on the use of AI in locating prior
      work.
\end{abstract}


% keywords can be removed
\keywords{Confusion Matrix \and Binary Classification \and Fowlkes--Mallows Index \and Matthews Correlation Coefficient \and F1}


\section{Introduction}

Evaluation of binary classifiers is central to the quantitative analysis of
machine learning models \cite{powers_evaluation_2011}.
%This paper will follow the notation in the Wikipedia article on the Confusion Matrix~\cite{wiki_cfsn} as of 2023-04-30.
Given a finite set of examples with known real labels, the quality of a set of
corresponding predicted labels can quantified using a $2 \times 2$ confusion
matrix.  A confusion matrix counts the number of true positives (\TP{}), true
negatives (\TN{}), false positives (\FP{}), and false negatives (\FN{}) a model
predicts with respect to the real labels. A confusion matrix is written as:

\begin{equation}
\begin{bmatrix}
    \TP{} & \FP{} \\
    \FN{} & \TN{} 
\end{bmatrix}	
\end{equation}

This matrix provides a holistic view of classifier quality, however, it is
often desirable to summarize performance using fewer numbers. Two popular
metrics defined on a classification matrix are precision and recall.

Precision --- also known as the positive-predictive-value (PPV) --- is the
fraction of positive predictions that are correct.

\begin{equation}
    \PPV{} = \frac{\TP{}}{\TP{} + \FP{}}
\end{equation}

Recall --- also known as the true positive rate (TPR), sensitivity, or
probability of detection (PD) --- is the fraction of real positive cases that
are correct.

\begin{equation}
    \TPR{} = \frac{\TP{}}{\TP{} + \FN{}}
\end{equation}

% https://en.wikipedia.org/wiki/F-score
One of the most popular confusion metrics is the F1 score. 
It can be defined as the harmonic mean of precision and recall.

\begin{equation}
    \Fone{} = \frac{2 \PPV{} \cdot \TPR{}}{\PPV{} + \TPR{}} = \frac{2 \TP{}}{2 \TP{} + \FP{} + \FN{}}
\end{equation}

% https://en.wikipedia.org/wiki/Fowlkes%E2%80%93Mallows_index
A similar, but less used metric is the Fowlkes-Mallows index~\cite{fowlkes_method_1983}, which was
  originally developed for measuring the similarity between two clusterings of a set of points.
It can be defined as the geometric mean of precision and recall~\cite{tharwat_classification_2020}.

\begin{equation}
    \Fowlkes{} = \sqrt{\PPV{} \cdot \TPR{}} = \sqrt{\frac{\TP{}}{\TP{} + \FP{}} \frac{\TP{}}{\TP{} + \FN{}}}
\end{equation}

In~\cite{powers_evaluation_2011}, Powers notes that the F1 score (and consequentially any metric that only
  includes precision and recall) only takes into account three of the four measures in a confusion matrix.
Powers, introduces modifications of precision and recall he refers to as informedness and markedness.
Additionally he advocates for the use of the MCC over the F1 measure.

The Matthews Correlation Coefficient (MCC)~\cite{matthews_comparison_1975} accounts for all four terms in the confusion matrix and is defined as:

\begin{equation}
    \MCC{} = \frac{%
        \TP{} \cdot \TN{} - \FP{} \cdot \FN{}
    }
    {\sqrt{%
        (\TP{} + \FP{}) (\TP{} + \FN{}) (\TN{} + \FP{}) (\TN{} + \FN{})
    }}
\end{equation}

While the MCC is a desirable measure due to its balanced inclusion of all terms in a confusion matrix, it
  requires that the number of true negatives is measurable.
In the case of object detection problems~\cite{zou2023object}, this is often intractable as the number of
  the number of predicted boxes and missed true boxes is dwarfed by the total number of boxes that the system
  correctly did not predict.
One can see this by considering the set of all $N\times M$ boxes centered at each pixel, most of which will
  be considered true negatives.
If the width and height of the boxes are allowed to extend outside the image, then the number of predictable
  boxes actually is unbounded (and even if they must be contained in the image, there will still be a very
  large number of them in real world cases).


Because calculating the number of true negatives is difficult for open-world problems like object detection,
  it is conceptually simpler to ignore true negatives and simply focus on the much smaller set of true
  positives, false positives, and false negatives, which can be used to compute PPV, TPR, F1, and FM.
While these measures have proven themselves to be effective, simply ignoring true negatives is somewhat
  unsatisfying.
We seek to remedy this noting that in these open-world problems the number of true negatives is so large it
  is effectively infinite and thus we ask the question:
what happens to the MCC as the number of true negatives approaches infinity?

The main contribution of this paper is to highlight a relationship between the MCC and the FM score:
the MCC reduces to FM as the number of true negatives approaches infinity.

% TODO: REVISE WORDING
\section{Related Work}
%\paragraph{Related work and terminology.}
In vegetation science and community ecology, $2\times 2$ contingency tables are commonly used to measure
association/similarity between two binary variables (e.g. the presence/absence of a species and membership
in a site group).
In that literature, the MCC is typically referred to as the \emph{phi coefficient} (or $\phi$ coefficient),
and the regime where true negatives become dominant is described as adding many \emph{double zeros}
(joint absences).
In this context, De C\'aceres et al.~\cite{decaceres_assessing_species_2008} compare the phi coefficient to its
limit value in large data sets, which they identify with the \emph{Ochiai index}.
De C\'aceres and Legendre~\cite{caceres_associations_species_2009} explicitly note that adding an infinite amount of
double zeros drives the correlation-form index to the Ochiai~(1957) similarity coefficient~\cite{ochiai_zoogeographic_studies_1957}.
Under the standard confusion-matrix identification $(a,b,c,d)=(\TP{},\FP{},\FN{},\TN{})$, the Ochiai
coefficient is
\[
    \mathrm{Ochiai} = \frac{\TP{}}{\sqrt{(\TP{}+\FP{})(\TP{}+\FN{})}}
    = \sqrt{\PPV{}\cdot\TPR{}},
\]
which is exactly the FM index.
Our aim is to make this relationship explicit in modern machine-learning notation, motivated by open-world
object detection, and to provide a mechanically checked formalization.
% /TODO: REVISE WORDING

Specifically, the contributions are:

\begin{itemize}

    \item We informally (but rigorously) prove the statement $\lim_{\TN{} \to \infty} \MCC{} = \Fowlkes$.

    \item We formally prove the statement $\lim_{\TN{} \to \infty} \MCC{} = \Fowlkes$ in Lean~4.

    \item We connect the ``double-zero'' discussion from ecology to the object detection
          setting, where the number of true negatives is effectively unbounded.

\end{itemize}



\section{The Relationship Between MCC and FM}%
\label{sec:headings}

\paragraph{Taking the limit of the MCC}

Let $\TP{}$, $\FP{}$, $\TN{}$, and $\FN{}$ be non-negative real numbers with
$\TP{} + \FP{} > 0$ and $\TP{} + \FN{} > 0$.  Consider the limit of the MCC as the
number of true negatives approaches infinity.

\begin{equation}
    \lim_{\TN{} \to \infty} \MCC{} = \lim_{\TN{} \to \infty}
    \frac{%
        \TP{} \cdot \TN{} - \FP{} \cdot \FN{}
    }
    {\sqrt{%
        (\TP{} + \FP{}) (\TP{} + \FN{}) (\TN{} + \FP{}) (\TN{} + \FN{})
    }}
\end{equation}

%%% Thanks to Lee Newberg for the cleaner formulation of this limit without L'Hôpital's rule!
We can take this limit by applying some algebra to the body of the limit. We multiply the numerator and denominator by $\frac{1}{\TN{}}$:

\begin{equation}\label{eq:lee_step1}
    = \lim_{\TN{} \to \infty}
    \frac{% 
        \frac{1}{\TN{}} (\TP{} \cdot \TN{} - \FP{} \cdot \FN{}) 
    }
    {\frac{1}{\TN{}} \sqrt{%
        (\TP{} + \FP{}) (\TP{} + \FN{}) (\TN{} + \FP{}) (\TN{} + \FN{})
    }} 
\end{equation}

We distribute the $\frac{1}{\TN{}}$ term in the numerator and denominator:

\begin{equation}\label{eq:lee_step2}
    = \lim_{\TN{} \to \infty}
    \frac{% 
        (\TP{} - \FP{} \cdot \frac{\FN{}}{\TN{}}) 
    }
    {\sqrt{%
        (\TP{} + \FP{}) (\TP{} + \FN{}) (\frac{\TN{} + \FP{}}{\TN{}}) (\frac{\TN{} + \FN{}}{\TN{}})
    }}
\end{equation}

The $\frac{\TN{}}{\TN{}}$ terms in the denominator cancel:

\begin{equation}\label{eq:lee_step3}
    = \lim_{\TN{} \to \infty}
    \frac{% 
        (\TP{} - \FP{} \cdot \frac{\FN{}}{\TN{}}) 
    }
    {\sqrt{%
        (\TP{} + \FP{}) (\TP{} + \FN{}) (1 + \frac{\FP{}}{\TN{}}) (1 + \frac{\FN{}}{\TN{}})
    }}
\end{equation}

The terms involving $\TN{}$ are fractions of simple rational polynomials
(w.r.t. $\TN{}$) and in each case the degree of the denominator is greater
than that of the numerator, so in the limit each of these terms simplifies to
$0$. Thus, the entire equation simplifies to:

\begin{equation}\label{eq:postlimit}
    = 
    \frac{% 
        (\TP{} - \FP{} \cdot 0) 
    }
    {\sqrt{%
        (\TP{} + \FP{}) (\TP{} + \FN{}) (1 + 0) (1 + 0)
    }}
\end{equation}

%We can take the limit of this equation using using L'Hôpital's rule --- i.e.
%the limit is equal to the limit of the derivative of the numerator with respect
%to \TN{} divided by the limit of the denominator with respect to \TN{}.

%\begin{equation}
%    = \lim_{\TN{} \to \infty}
%    \frac{
%        \frac{\partial \TP{} \cdot \TN{} - \FP{} \cdot \FN{}}{\partial \TN{}} 
%    }
%    {
%        \frac{\partial \sqrt{(\TP{} + \FP{}) (\TP{} + \FN{}) (\TN{} + \FP{}) (\TN{} + \FN{})}}{\partial \TN{}}
%    } 
%\end{equation}

%The derivative of the numerator simplifies to $\TP{}$ by applying the product
%rule. The derivative of the denominator can be taken using the chain rule and
%product rules. Explicitly showing this process is not too difficult, but it is
%involved. For conciseness we omit the explicit steps and compute the derivative
%symbolically using a computer. Taking the derivative of the numerator and
%denominator we get:

%\begin{equation}
%    = \lim_{\TN{} \to \infty}
%    \frac{
%        \TP{} 
%    }
%    {
%        \sqrt{
%            \frac{
%                (\FN{} + \TN{}) (\FN{} + \TP{}) (\FP{} + \TP{})
%            }{
%                4 (\FP{} + \TN{})
%            }} + 
%        \sqrt{
%            \frac{
%                (\FP{} + \TN{}) (\FN{} + \TP{}) (\FP{} + \TP{})
%            }{
%                4 (\FN{} + \TN{})
%            }
%        }
%    }
%\end{equation}

%In the limit the terms with \TN{} cancel, resulting in:


%\begin{equation}
%    =
%    \frac{
%        \TP{} 
%    }
%    {
%        \frac{
%            \sqrt{(\FN{} + \TP{}) (\FP{} + \TP{})}
%        }{
%            2 
%        } + 
%        \frac{
%            \sqrt{(\FN{} + \TP{}) (\FP{} + \TP{})}
%        }{
%            2 
%        }
%    } 
%\end{equation}

Thus we find that the limit of
the MCC as true negatives approach infinity is:

\begin{equation}
    = 
    \frac{\TP{}}
    {\sqrt{%
        (\TP{} + \FP{}) (\TP{} + \FN{}) 
    }}
\end{equation}

\paragraph{Rearranging the FM}

Now rearranging the equation for FM, we find it is equivalent to the limit of the MCC as the number of true negatives approaches infinity.

\begin{align}
    \Fowlkes{} &= \sqrt{\frac{\TP{}}{\TP{} + \FP{}} \frac{\TP{}}{\TP{} + \FN{}}} \\
               &= \sqrt{\frac{\TP{}^2}{(\TP{} + \FP{}) (\TP{} + \FN{})}} \\
               &= \frac{\TP{}}{\sqrt{(\TP{} + \FP{}) (\TP{} + \FN{})}} \\
               &= \lim_{\TN{} \to \infty} \MCC{}
\end{align}
%\end{equation}

\paragraph{Verifying the proof}

The correctness of these claims can be verified using SymPy~\cite{sympy17}. We
define a symbolic expression for the definition of the MCC and FM score. We
then use SymPy to determine the limit of the MCC as $\TN{} \to \infty$. 
Finally we subtract expressions that we claim are equal, which will result in
zero only if they are equal.

\begin{lstlisting}[language=Python]

from sympy import sqrt, symbols, simplify
from sympy.series import limit

tp, tn, fp, fn = symbols("tp tn fp fn",
                         integer=True, negative=False)

# The definition of the MCC
numer = (tp * tn - fp * fn)
denom = sqrt((tp + fp) * (tp + fn) * (tn + fp) * (tn + fn))
mcc = numer / denom

# The definition of FM
FM = sqrt((tp / (tp + fn)) * (tp / (tp + fp)))

# Compute the limit of the MCC definition
mcc_lim = limit(mcc, tn, float("inf"))

# We claim the limit of the MCC and the FM are equivalant to:
mcc_lim_claim = tp / sqrt((tp + fn) * ((tp + fp)))

# Check the claim is equal to FM
assert simplify(FM - mcc_lim_claim) == 0
# Check the claim is equal to the MCC limit
assert simplify(mcc_lim - mcc_lim_claim) == 0

\end{lstlisting}

The above program does not raise an AssertionError, thus we have proven
$\lim_{\TN{} \to \infty} \MCC{} = \Fowlkes{}, \square{}$.

\section{Formal Verification with Lean}%
\label{sec:lean}

This section is an addendum. The original body of this paper was written on
2023--07--11; the Lean formalization and this section were added on 2025--11--29.

We formalized the main result in Lean~4~\cite{moura2021lean}, an interactive
theorem prover with a small trusted kernel.  A Lean proof provides a
machine-checked guarantee that each algebraic manipulation and limit argument
used in the informal derivation is valid under explicit assumptions.

The proof is small enough that we can present it here, heavily commented to make the structure clear for
  readers who are unfamiliar with Lean.
It is possible to dramatically simplify the proof, but this variant is better aligned with the arguments in
  \Cref{sec:headings}.

%The main theorem is expressed as a limit along \code{atTop} on the true-negative
%coordinate:

\noindent

\definecolor{commentcolor}{rgb}{0.0,0.5,0.0}
\definecolor{stringcolor}{rgb}{0.58,0.0,0.82}
\definecolor{numbercolor}{rgb}{0.5,0.5,0.5}
\definecolor{tacticcolor}{rgb}{0.58,0.0,0.82}

%\def\lstlanguagefiles{lstlean.tex}

\lstdefinelanguage{lean}{%
  sensitive=true,
  morecomment=[l]{--},
  morecomment=[s]{/-}{-/},
  %morestring=[b]",
  % “structure” keywords
  morekeywords=[1]{import,open,namespace,section,end,noncomputable,variable,variables,
    theorem,lemma,def,abbrev,example,instance,structure,class,inductive,notation,
    by,where,let,have,show,fun,match,with,if,then,else,do,return,in,forall},
  % tactics / common commands (tune this list)
  morekeywords=[2]{simp,simpa,rw,erw,dsimp,unfold,ring,ring_nf,nlinarith,linarith,
    positivity,norm_num,field_simp,omega,tauto,exact,refine,apply,assumption,
    intro,intros,ext,funext,cases,rcases,obtain,constructor,left,right,
    classical,by_contra,contrapose,finish,taut,done,convert,using,suffices,calc},
}

\newcommand{\surdSym}{\ensuremath{\raisebox{0.3ex}{\scalebox{0.8}{$\surd$}}}}

\lstdefinestyle{leanpretty}{%
    language=lean,
    columns=fullflexible,
    backgroundcolor=\color{backcolour},   
    commentstyle=\color{codegreen},
    keywordstyle=\color{keywordcolor},
    numberstyle=\tiny\color{codegray},
    stringstyle=\color{codepurple},
    basicstyle=\ttfamily\footnotesize,
    breakatwhitespace=false,         
    breaklines=true,                 
    captionpos=b,                    
    keepspaces=true,                 
    numbers=left,                    
    numbersep=5pt,                  
    showspaces=false,                
    showstringspaces=false,
    showtabs=false,                  
    tabsize=2,
    literate=
      {`}{\textasciigrave}1
      {\\Real}{$\mathbb{R}$}5
      {\\le}{$\le$}3
      {\\ge}{$\ge$}3
      {\\ne}{$\ne$}3
      {\\to}{$\to$}3
      {\\in}{$\in$}3
      {\\nhdsSym}{$\mathcal{N}$}8
      {\\Infinity}{$\infty$}9
      {\\root}{\surdSym}5
      {\\eqF}{$=\raisebox{-0.3ex}{^{\mathrm{f}}}$}4
}

% python ~/code/paper-g1-and-mcc/map_lean_unicode.py --emit-literate
% Use it:
\lstset{style=leanpretty}

%\input{leanproof_v2.tex}

First we import basic mathlib functionality and define the terms used in the proof.

\begin{lstlisting}[language=lean]
import Mathlib.Tactic
open Filter Topology

noncomputable section

/-- Precision (positive predictive value) -/
def PPV (TP FP : \Real) : \Real := TP / (TP + FP)

/-- Recall (true positive rate) -/
def TPR (TP FN : \Real) : \Real := TP / (TP + FN)

/-- Fowlkes–Mallows index -/
def FM (TP FP FN : \Real) : \Real := \root (PPV TP FP * TPR TP FN)

/-- Matthews correlation coefficient -/
def MCC (TP TN FP FN : \Real) : \Real :=
  (TP * TN - FP * FN) / \root ((TP + FP) * (TP + FN) * (TN + FP) * (TN + FN))
\end{lstlisting}

Next, we define several lemmas that prove generic facts that will be useful in
the main theorem.

\begin{lstlisting}[language=lean]
/-- The basic fact `c / x \to 0` as `x \to +\Infinity`.
Limits in Lean are expressed with filters.
`Tendsto f atTop (\nhdsSym L)` is the Lean form of `lim_{x \to +\Infinity} f x = L`. -/
lemma tendsto_const_div_atTop_nhds_0 (c : \Real) :
    Tendsto (fun x : \Real => c / x) atTop (\nhdsSym 0) :=
  tendsto_const_nhds.div_atTop Filter.tendsto_id

/-- If `c/x \to 0` then `1 + c/x \to 1` (limit rule for addition). -/
lemma tendsto_one_add_const_div_atTop (c : \Real) :
    Tendsto (fun x : \Real => (1 : \Real) + c / x) atTop (\nhdsSym 1) := by
  simpa using (tendsto_const_nhds.add (tendsto_const_div_atTop_nhds_0 c))

/-- If `c/x \to 0` then `a - c/x \to a` (limit rule for subtraction). -/
lemma tendsto_const_sub_const_div_atTop (a c : \Real) :
    Tendsto (fun x : \Real => a - c / x) atTop (\nhdsSym a) := by
  simpa using (tendsto_const_nhds.sub (tendsto_const_div_atTop_nhds_0 c))

/- A common pattern we will use: `A * (1 + c/x) * (1 + d/x) \to A`.
This is just the limit rule for multiplication, plus the fact that constants tend to
themselves, and `(1 + c/x) \to 1`.-/
lemma tendsto_const_mul_one_add_mul_one_add_div_atTop (A c d : \Real) :
    Tendsto (fun x : \Real => A * (1 + c / x) * (1 + d / x)) atTop (\nhdsSym A) := by
  have := (tendsto_one_add_const_div_atTop c).mul (tendsto_one_add_const_div_atTop d)
  simpa [mul_assoc] using (tendsto_const_nhds.mul this)

/-- If `a > 0` then `sqrt(a) \ne 0` (since `sqrt(a) > 0`).
In Lean, a theorem or lemma is stated in the context of named hypotheses
(assumptions). Read this as: given the condition `h_agt0` the following claim is true. -/
lemma sqrt_of_pos_ne_zero {a : \Real} (h_agt0 : 0 < a) : \root a \ne 0 :=
  ne_of_gt (Real.sqrt_pos.mpr h_agt0)

/- A generic algebraic step used in the MCC denominator manipulation.
`sqrt(x) / t = sqrt(x / t^2)`  (assuming `0 \le x` and `0 \le t`).
`aesop` (Automated Extensible Search for Obvious Proofs) is an automation tactic.
It performs a small proof search using simp rules and standard lemmas. -/
lemma sqrt_div_eq_sqrt_div_sq {x t : \Real} (h_xge0 : 0 \le x) (h_tge0 : 0 \le t) :
    \root x / t = \root (x / (t ^ 2)) := by aesop
\end{lstlisting}

We are now prepared to state and prove the main theorem. Under conditions where
confusion matrix entries are non-zero, and denominators are non-zero, the
Matthews Correlation Coefficient converges towards the Fowlkes-Mallows index as
true negatives become arbitrarily large:

\begin{lstlisting}[language=lean]
theorem tendsto_MCC_atTop_eq_FM
    {TP FP FN : \Real}
    (hTPFPpos : 0 < TP + FP)
    (hTPFNpos : 0 < TP + FN)
    (hTP_geq0 : 0 \le TP)
    (hFP_geq0 : 0 \le FP)
    (hFN_geq0 : 0 \le FN) :
    Tendsto (fun TN : \Real => MCC TP TN FP FN) atTop (\nhdsSym (FM TP FP FN)) := by
  -- `A` is the constant factor that does not depend on TN.
  let A : \Real := (TP + FP) * (TP + FN)
  -- This is the “post step 3” expression: the same one that appears in the algebraic limit.
  let post_step3 : \Real \to \Real := fun TN =>
    (TP - FP * FN / TN) / \root (A * (1 + FP / TN) * (1 + FN / TN))
  ----------------------------------------------------------------------
  -- Step 1/2/3 algebraic rewrite: for TN > 0, MCC(TN) = post_step3(TN).
  ----------------------------------------------------------------------
  -- `f \eqF[atTop] g` means: `f TN` eventually equals `g TN` for all sufficiently large `TN`.
  have h_steps_123 :
      (fun TN : \Real => MCC TP TN FP FN) \eqF[atTop] post_step3 := by
    -- `filter_upwards` is a convenient way to work with “eventually” statements.
    -- Here it gives us an arbitrary `TN` with the assumption `0 < TN`.
    filter_upwards [Filter.eventually_gt_atTop (0 : \Real)] with TN hTN_gt0

    -- After unfolding, the goal is a concrete identity between real expressions.
    simp [MCC, post_step3, A]
    -- Step 2: distribute the factor `1/TN` into the numerator.
    have h_num : (TP * TN - FP * FN) / TN = TP - FP * FN / TN := by
      field_simp [hTN_gt0.ne'] 
    -- Step 3: rewrite `(TN + FP)/TN` as `1 + FP/TN`, similarly for `FN`.
    have h_inside :
        ((TP + FP) * (TP + FN) * (TN + FP) * (TN + FN)) / (TN ^ 2) =
          (TP + FP) * (TP + FN) * (1 + FP / TN) * (1 + FN / TN) := by field_simp [hTN_gt0.ne']
    -- Name the large product under the MCC square root.
    let mcc_inside_denom : \Real := (TP + FP) * (TP + FN) * (TN + FP) * (TN + FN)
    -- Side condition needed to move division under `sqrt`.
    have hDenomGe0 : 0 \le mcc_inside_denom := by simp [mcc_inside_denom]; positivity
    -- Denominator rewrite: push `/TN` inside `sqrt`, then substitute the Step-3 identity.
    have h_sqrt :
        \root ((TP + FP) * (TP + FN) * (TN + FP) * (TN + FN)) / TN =
          \root (A * (1 + FP / TN) * (1 + FN / TN)) := by
      simpa [mcc_inside_denom, A] using
        (sqrt_div_eq_sqrt_div_sq (x := mcc_inside_denom) (t := TN) hDenomGe0 hTN_gt0.le).trans
          (by aesop)
    -- Final algebraic combination: divide numerator and denominator by TN.
    calc
      (TP * TN - FP * FN) / \root ((TP + FP) * (TP + FN) * (TN + FP) * (TN + FN)) =
        ((TP * TN - FP * FN) / TN) /
          (\root ((TP + FP) * (TP + FN) * (TN + FP) * (TN + FN)) / TN) := by field_simp [hTN_gt0.ne']
      _ = (TP - FP * FN / TN) / \root (A * (1 + FP / TN) * (1 + FN / TN)) := by aesop
  ----------------------------------------------------------------------
  -- Limit of post_step3: as TN \to +\Infinity, the small fractions FP/TN and FN/TN go to 0.
  ----------------------------------------------------------------------
  -- Numerator limit: `TP - (FP*FN)/TN \to TP`.
  have h_num_lim :
      Tendsto (fun TN : \Real => TP - FP * FN / TN) atTop (\nhdsSym TP) :=
    tendsto_const_sub_const_div_atTop TP (FP * FN)
  -- Denominator limit: `sqrt(A * (1 + FP/TN) * (1 + FN/TN)) \to sqrt(A)`.
  have h_den_lim :
      Tendsto (fun TN : \Real => \root (A * (1 + FP / TN) * (1 + FN / TN))) atTop (\nhdsSym (\root A)) :=
      by simpa using
       (Filter.Tendsto.sqrt (tendsto_const_mul_one_add_mul_one_add_div_atTop A FP FN))
  -- The quotient limit rule needs the limit denominator to be nonzero.
  have h_den_ne : \root A \ne 0 := by
     have hApos : 0 < A := by simpa [A] using (mul_pos hTPFPpos hTPFNpos)
     exact sqrt_of_pos_ne_zero hApos
  -- Quotient limit rule: if num \to Num and den \to Den with Den \ne 0, then num/den \to Num/Den.
  have h_post_lim : Tendsto post_step3 atTop (\nhdsSym (TP / \root A)) := by
    exact Filter.Tendsto.div h_num_lim h_den_lim h_den_ne
  -- Transfer the limit from `post_step3` back to `MCC` using eventual equality.
  have h_mcc_lim :
      Tendsto (fun TN : \Real => MCC TP TN FP FN) atTop (\nhdsSym (TP / \root A)) := by
    exact h_post_lim.congr' (Filter.EventuallyEq.symm h_steps_123)
  ----------------------------------------------------------------------
  -- Rewrite FM into the same closed form `TP / sqrt(A)`.
  ----------------------------------------------------------------------
  have h_FM : FM TP FP FN = TP / \root A := by
    have: TP / (TP + FP) * (TP / (TP + FN)) = (TP ^ 2) / ((TP + FP) * (TP + FN)) := by
      field_simp [hTPFPpos.ne', hTPFNpos.ne']
    simp_all [FM, PPV, TPR, A]
  /- `h_mcc_lim` shows the limit is `TP / sqrt A`. `h_FM` shows `FM` can be rewritten as the same
  value. Substituting this rewrite into the target of `h_mcc_lim` completes the proof. -/
  simpa [h_FM] using h_mcc_lim
\end{lstlisting}


The above Lean code compiles with version
\texttt{leanprover/lean4:v4.28.0-rc1}, thus --- up to soundness holes --- we
have proven: $\lim_{\TN{} \to \infty} \MCC{} = \Fowlkes{}, \square{}$.

%Here \code{MCC} and \code{FM} are the direct Lean definitions corresponding to
%the formulas in this paper. The full proof is 654 lines. It is available in
%these forms:

%https://github.com/Erotemic/paper-g1-and-mcc/blob/main/mcc-proof/MCC_atTop_eq_FM/WuAristotle.lean
%\begin{itemize}
%\item GitHub permalink: \\
%\href{https://github.com/Erotemic/paper-g1-and-mcc/blob/89dc05e80432f727cc41f07ecfeda5c87e69945d/mcc-proof/MCC_atTop_eq_FM/WuAristotle.lean}{%
%    \texttt{https://github.com/Erotemic/paper-g1-and-mcc/blob/}\\%
%    \texttt{89dc05e80432f727cc41f07ecfeda5c87e69945d/}%
%\texttt{mcc-proof/MCC\_atTop\_eq\_FM/WuAristotle.lean}}

%%** TODO: Add sha256 checksum **

%\item IPFS snapshot (CID): \\
%\href{https://ipfs.io/ipfs/bafybeihfg46sgh2bg2fy3rph5wjdzgwcrxufph2w6vd35isz2rhbvvibpu}{%
%\texttt{https://ipfs.io/ipfs/}%
%\texttt{bafybeihfg46sgh2bg2fy3rph5wjdzgwcrxufph2w6vd35isz2rhbvvibpu}}
%\end{itemize}

%%# Aristotle version
%%bafybeihfg46sgh2bg2fy3rph5wjdzgwcrxufph2w6vd35isz2rhbvvibpu

%%# GPT 5.1 version
%%bafkreihwpqlr3s2mcu6csopndx3pvpo5az7m66iw5ox4lnqk7i3jjihnky


%The GitHub permalink fixes the exact commit, and the IPFS CID fixes the exact
%file content, providing redundancy against link rot and build-in checksum
%verification.

\section{AI Usage}%
\label{sec:ai_usage}

AI has begun to transform math and science research. 
There are several notable experiences the author would like to mention.

\paragraph{Construction with large language models}
The Lean proof was assembled interactively with assistance from the GPT-5.1
  model~\cite{openai_gpt5_system_card}.
Earlier attempts using the GPT-5 and GPT-4o models~\cite{openai_gpt4o_system_card} were not successful in
  producing working Lean proof.
GPT-5.1 was able to complete the proof in 654 lines, but it manually required feedback from lean and was
  prone to syntax errors.
Further refinement from Aristotle~\cite{achim2025aristotleimolevelautomatedtheorem} and users on
  leanprover.zulipchat~\cite{crall2025zulip561031593} refined this to 66 lines (including comments).
Building on that, GPT-5.2\cite{openai_gpt52_system_card} with extended thinking and manual feedback was able
  to produce the proof that matches the structure or the original natural language argument, and is the
  version presented here.
% Note aristotle proof

\paragraph{Note on discovery and terminology drift.}
The relationship proved in this paper was initially investigated by the author under the terminology of
binary-classification evaluation (MCC, precision/recall, and the Fowlkes--Mallows score) and motivated by
open-world object detection, where the number of true negatives is ill-defined or effectively unbounded.
While searching within this vocabulary did not reveal an explicit prior statement of the limit,
we later found closely related discussion in the ecology literature expressed using different names
for the same $2\times 2$ contingency table quantities (phi coefficient, double zeros, and the Ochiai similarity).
Modern AI-assisted semantic search was instrumental in suggesting this alternate terminology, after which the
connection was verified directly by reading the cited sources and translating notation.


\section{Conclusion}

This paper proves that the limit of the MCC as the number of true negatives
goes to infinity is equivalent to the Fowlkes--Mallows index (i.e. the
geometric mean of precision and recall).

This is a useful insight in open world problems where the number of true negative cases grows faster than
  the number of other confusion categories.
It validates the use of precision and recall as a way of describing the quality of object detection results
  and hints that the FM score may be a preferable alternative to the more standard F1 score.


\section{Acknowledgements}

Thanks to Lee Newberg for the cleaner formulation of \cref{eq:lee_step1,eq:lee_step2,eq:lee_step3} to achieve \cref{eq:postlimit} without needing L'Hôpital's rule.

\bibliographystyle{unsrtnat}
\bibliography{references}  %%% Uncomment this line and comment out the ``thebibliography'' section below to use the external .bib file (using bibtex) .

\end{document}

